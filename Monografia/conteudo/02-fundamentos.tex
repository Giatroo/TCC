%!TeX root=../tese.tex

%%%%%%%%%%%%%%%%%%%%%%%%%%%%%%%%%%%%%%%%%%%%%%%%%%%%%%%%%%%%%%%%%%%%%%%%%%%%%%%%

\chapter{Fundamentos e Trabalhos Relacionados}%
\label{cha:fundamentos_e_trabalhos_relacionados}

\section{Detecção de Sarcasmo}%
\label{sec:deteccao_de_sarcasmo}

De acordo com o dicionário Dicio, sarcasmo é uma zombaria que busca ofender,
enquanto ironia é a ação de dizer o oposto do que se deseja expressar. Ainda
segundo ele, a diferença entre esses dois termos se dá no fato de que sarcasmo é
um dito ácido que pode ou não ser expresso por meio de uma ironia e essa, por
sua vez, pode ou não ser utilizada para
ofender.~\cite{dicio_sarc, dicio_irony}

\cite{sarcasm:survey}

(R. Giora, On Irony and Negation)

(H. Paul Grice. 1975. Logic and Conversation)

(Irony and Sarcasm: Corpus Generation and Analysis Using Crowdsourcin)

O termo detecção de sarcasmo refere-se à determinação de se há ou não sarcasmo
em uma porção de texto verbal. E o termo detecção automática de sarcasmo
refere-se a métodos computacionais de se resolver o problema acima mencionado.
Computacionalmente, podemos definir esse problema como uma
\textit{\textbf{classificação binária}}, termo explicado mais a frente.

Entretanto, na literatura é bastante comum também se definir detecção de
sarcasmo como a determinação de se há ou não sarcasmo ou ironia verbal em uma
porção de texto verbal. Portanto, em geral, ao se falar de sarcasmo, a
literatura engloba tanto sarcasmo quanto ironia como se fossem a mesma coisa.
Este texto também não fará discriminação entre sarcasmo ou ironia.

Em geral, esse problema é difícil, pois, por vezes, nem humanos conseguem
perceber essa figura de linguagem (e.g. ``\textit{Oba, hoje está tão ensolarado,
que vontade de ir para a escola.}''). Além disso, o contexto tende a importar
muito. Muitas características externas ao texto podem servir para descriminar se
uma pessoa está ou não sendo sarcástica. Entre alguns exemplos estão intonação,
locutor, interlocutor, conhecimento prévio sobre a fala, tempo e espaço em que
se fala e elementos não verbais.

(Humans Require Context to Infer Ironic Intent (so Computers Probably do, too))

\section{Classificação binárias}%
\label{sec:classificacao_binarias}

O problema de detecção de sarcasmo pode ser tratado como uma classificação
binária. Esse tipo de problema é muito estudado no campo do aprendizado de
máquina e envolve classificar os elementos de um conjunto em dois grupos
chamados de classes. Por classificar entende-se a determinação de um elemento do
conjunto entre pertencente à primeira ou segunda classe.

No caso da detecção de sarcasmo, os elementos são os pedaços de texto e as duas
classes são \textit{ser sarcástico} e \textit{não ser sarcástico}. Em termos
matemáticos, tem-se um conjunto de exemplos denotado pela matriz $X$ de
dimensões $n\times m$, onde $n$ é o número de exemplos e $m$ é o número de
características que se usa para descrever cada um desses exemplos. Cada linha de
$X$ é denotada por $\xii[x]{i}$ e chamada de \textit{exemplo} $i$ ou
\textit{instância} $i$.

A linha $\xii[x]{i}$ é um vetor de $m$ posições em que cada posição é uma
característica que descreve a entrada. Por exemplo, na detecção de sarcasmo, o
primeiro valor pode ser a quantidade de palavras lidas, o segundo valor pode ser
a quantidade de caracteres, a terceira pode ser a soma de positividade do texto
(definida por algum critério específico) e assim por diante. É importante que a
mesma posição em diferentes instâncias tenha o mesmo significado. Portanto, caso
se defina que a primeira posição represente o número de palavras, todas as
instâncias devem seguir essa regra.

Cada instância $i$ pertence ou à primeira classe ou à segunda, modela-se isso
por um valor $\xii[y]{i}$ chamado de $i$-ésimo rótulo (do inglês,
\textit{label}). E escreve-se $\xii[y]{i}\in\set{0, 1}$, onde cada valor
representa uma das classes. No caso de sarcasmo, pode-se representar por
\textit{não sarcasmo} o valor $0$ e \textit{sarcasmo}, o valor $1$. Com esses
valores $\xii[y]{i}$ constrói-se um vetor $y$ onde $y_i=\xii[y]{i}$.

Ao final, o problema é descobrir uma função $f$ que mapeie bem $X$ em $y$. Note,
entretanto, que $n$, o número de exemplos, é possivelmente infinito, pois sempre
se pode achar novos exemplos e testar se $f$ os mapeia corretamente. Ou seja,
tenta-se achar essa função $f$ conhecendo parcialmente o conjunto das
instâncias.

\subsection{Métricas de avaliação}%
\label{sub:metricas_de_avaliacao}

Para saber se a função $f$ mapeia bem $X$ em $y$, é preciso definir uma métrica
de avaliação, que é uma função matemática bem definida que indica quanto de erro
se comete em determinado conjunto $X$.

Dado um conjunto de entrada $X$ com $n$ instâncias, seja $y$ o vetor que
representa se cada instância é ou não sarcástica. Seja então $f$ a função sob
avaliação (ou seja, a função de modelagem que recebe os textos e determina se
são ou não sarcásticos). Seja $\hat{y}=f(X)$ os resultados gerados por essa
função.

Note que para cada exemplo $i$, podemos ter duas opções:
\begin{enumerate}
  \item $\hat{y}_i=y_i$  (\textit{acerto})
  \item $\hat{y}_i\neq y_i$ (\textit{erro})
\end{enumerate}

Além disso, pode-se note que o par $\pair{y_i}{\hat{y}_i}$ pode ter quatro
valores:
\begin{enumerate}
  \item $\pair{1}{1}$, caso \textit{verdadeiro positivo} (também conhecido como
    \textit{true positive} e denotado por $tp$).
  \item $\pair{0}{1}$, caso \textit{falso positivo} (também conhecido como
    \textit{false positive} e denotado por $fp$).
  \item $\pair{0}{0}$, caso \textit{verdadeiro negativo} (também conhecido como
    \textit{true negative} e denotado por $tn$).
  \item $\pair{1}{0}$, caso \textit{falso negativo} (também conhecido como
    \textit{false negative} e denotado por $fn$).
\end{enumerate}

Definidos esses valores, pode-se definir algumas métricas que ajudam a perceber
quão boa a função $f$ é. Dessa forma, pode-se comparar duas funções $f$.

\paragraph{Acurácia}%
\label{par:acuracia}

Definida simplesmente como a quantidade total de acertos divido pela quantidade
total de instâncias. Seja $n'$ a quantidade de acertos, então:
\[
\texttt{Acurácia} = \dfrac{n'}{n}
\]

É comum também definir a acurácia em termos dos vamos acima listados:
\[
\texttt{Acurácia} = \dfrac{tp+tn}{tp+tn+fp+fn}
\]

Note que $tp+tn$ representa justamente quando $\hat{y}_i=y_i$ e $fp+fn$,
$\hat{y}_i\neq y_i$.

\paragraph{Precisão}%
\label{par:precisao}

Definida como:
\[
\texttt{Precisão} = \dfrac{tp}{tp+fp}
\]

é a taxa de verdadeiros positivos em relação a todos os positivos. Portanto,
mede a fração de quantos positivos foram acertados em relação a todos os que
existiam no conjunto observado.

\paragraph{Revocação}%
\label{par:revocacao}

Muitas vezes chamada por seu nome em inglês, \textit{recall}, é definida como:
\[
\texttt{Revocação} = \dfrac{tp}{tp+fn}
\]

é a taxa de verdadeiros positivos em relação todas as instâncias preditas como
positivas. Portanto, a fração entre todos as instâncias previstas como positivas
e as instâncias que realmente eram positivas.

A precisão e a revocação guardam uma relação chave entre si e geralmente se quer
ter um bom balanço entre as duas. Observe essa relação a partir do seguinte
exemplo.

Comece com a função
\begin{equation}
  f\paren{\xii[x]{i}} = 1~~\forall i
\end{equation}

Como ela sempre considera a entrada como positiva, então os únicos casos
existentes serão de verdadeiros positivos e falso positivos. Logo:
\[
\texttt{Precisão} = \dfrac{tp}{tp+fp} = \dfrac{tp}{tp+(n-tp)} = \dfrac{tp}{n}
\]
e
\[
\texttt{Revocação} = \dfrac{tp}{tp+fn} = \dfrac{tp}{tp+0} = 1
\]

Portanto, a revocação terá o maior valor possível, enquanto a precisão será
exatamente a fração de valores positivos que o conjunto observado possui. Como,
semanticamente, a revocação é penalizada sempre que um elemento é positivo, mas
foi previsto como negativo, então faz sentido que se tenha uma revocação de
$100\%$, já que não se comete esse tipo de erro.

Agora, tome a função
\begin{equation}
  f\paren{\xii[x]{i}} = 0~~\forall i
\end{equation}

Como ela sempre considera a entrada como negativa, então os únicos casos
existentes serão de verdadeiro negativo e falso negativo. Logo:
\[
\texttt{Precisão} = \dfrac{tp}{tp+fp} = \dfrac{0}{0+fp} = 0
\]
e
\[
\texttt{Revocação} = \dfrac{tp}{tp+fn} = \dfrac{0}{0+fn} = 0
\]

Portanto, como a função nunca tenta marcar uma instância como positiva, ela
nunca acerta os verdadeiros positivos e comete $100\%$ de erro.

\paragraph{Métrica $F_1$}%
\label{par:metrica_f_1_}

Por esses e outros motivos, é interessante agregar a precisão e revocação em um
único valor, que permite fácil comparação entre dois modelos. É possível
utilizar uma média simples desses dois valores, mas é muito mais comum a
utilização da métrica $F_1$ (ou, do inglês, $F_1$ \textit{score}).

Ela é definida como:
\[
F_1=2\cdot
\dfrac{\texttt{precisão}\cdot\texttt{revocação}}
{\texttt{precisão}+\texttt{revocação}}
\]

\section{Trabalhos Relacionados}%
\label{sec:trabalhos_relacionados}

\subsection{Abordagens Baseadas em Regras}%
\label{sub:abordagens_baseados_em_regras}

Abordagens baseadas em regras são aquelas que usam regras fixas para determinar
se uma sequência de palavras contém ou não ironia. Para criar essas regras,
várias características do texto podem ser utilizadas, como as classes sintáticas
das palavras, se são palavras de cunho positivo ou negativo, se há a presença ou
não de certas palavras em uma ordem, entre qualquer outro tipo de regra
imaginável.

Essa abordagem é computacional, porque as regras formam um algoritmo que pode
ser implementado por uma linguagem de computação. Dessa forma, qualquer
sequência de palavras pode ser dada como entrada para o algoritmo e ele
retornará se ele acha que essa sequência contém ou não sarcasmo.

\cite{veale:2010} investigam em seu artigo sequências da forma ``\textit{as * as
a *''} (``tão * quanto um(a) *''), consideradas como analogias entre o que os
autores chamam de \textit{base} (\textit{ground}) e \textit{meio}
(\textit{vehicle}). Eles utilizam a API do Google para coletar 45021 instâncias
do padrão ``\textit{about as * as *}'' e filtram os resultados na mão para
chegar em 20299 instâncias que de fato são consideradas analogias. Eles então
anotam manualmente os rótulos para essas instâncias e encontraram que 15502
casos ($76\%$) são irônicos e apenas 4797 ($24\%$) são não irônicos.

Então, dado uma sequência ``\textit{as * as a *}'', os autores utilizam
mecanismos de busca na rede como a API do Google para distinguir entre três
casos: os casos em que essa sequência nunca foi usada como ``\textit{about}'';
aqueles que já foram, mas não frequentemente; e aqueles que que são
frequentemente usados com essa marcação. Segundo os próprios autores, essas três
categorias proveem, respectivamente, evidencia fraca contra ironia, evidencia
fraca a favor da ironia e evidência forte para ironia. Ou seja, o caso em que
temos mais certeza de que é uma ironia é o caso em que a sequência é
frequentemente utilizada junto com a palavra ``\textit{about}''.

Assim sendo, \cite{veale:2010} criam um sequência de nove passos para
classificar uma frase desse tipo entre as classes \textit{ironia} e \textit{não
ironia}. Esses passos são bastante claros e precisos, podendo ser, portanto,
implementados por um programa de computador e caracterizando, por conseguinte,
uma detecção automática de sarcasmo.

Nessas regras, os autores utilizam o fato de que analogias mais frequentemente
utilizadas são menos prováveis de serem irônicas, pois a ironia é utiliza
bastante a criatividade do locutor para fazer analogias não usuais.

Devido à natureza das regras propostas pelos autores, eles conseguem medir a
precisão e revocação obtida por cada uma das regras. No geral, o modelo atinge
uma acurácia de $88\%$, o que é um valor bastante alto para esse tipo de
problema. Entretanto, note-se que os autores se limitaram a um escopo muito
fechado (das frases do tipo ``\textit{as * as a *}'').
