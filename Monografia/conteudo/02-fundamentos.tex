%!TeX root=../tese.tex

%%%%%%%%%%%%%%%%%%%%%%%%%%%%%%%%%%%%%%%%%%%%%%%%%%%%%%%%%%%%%%%%%%%%%%%%%%%%%%%%

\chapter{Fundamentos e Trabalhos Relacionados}%
\label{cha:fundamentos_e_trabalhos_relacionados}

\section{Detecção de Sarcasmo}%
\label{sec:deteccao_de_sarcasmo}

De acordo com o dicionário Dicio, sarcasmo é uma zombaria que busca ofender,
enquanto ironia é a ação de dizer o oposto do que se deseja expressar. Ainda
segundo ele, a diferença entre esses dois termos se dá no fato de que sarcasmo é
um dito ácido que pode ou não ser expresso por meio de uma ironia e essa, por
sua vez, pode ou não ser utilizada para
ofender.\cite{dicio_sarc}\cite{dicio_irony}

(H. Paul Grice. 1975. Logic and Conversation)

(Irony and Sarcasm: Corpus Generation and Analysis Using Crowdsourcin)

O termo detecção de sarcasmo refere-se à determinação de se há ou não sarcasmo
em uma porção de texto verbal. E o termo detecção automática de sarcasmo
refere-se a métodos computacionais de se resolver o problema acima mencionado.
Computacionalmente, podemos definir esse problema como uma
\textit{\textbf{classificação binária}}, termo explicado mais a frente.

Entretanto, na literatura é bastante comum também se definir detecção de
sarcasmo como a determinação de se há ou não sarcasmo ou ironia verbal em uma
porção de texto verbal. Portanto, em geral, ao se falar de sarcasmo, a
literatura engloba tanto sarcasmo quanto ironia como se fossem a mesma coisa.
Este texto também não fará discriminação entre sarcasmo ou ironia.

Em geral, esse problema é difícil, pois, por vezes, nem humanos conseguem
perceber essa figura de linguagem (e.g. ``\textit{Oba, hoje está tão ensolarado,
que vontade de ir para a escola.}''). Além disso, o contexto tende a importar
muito. Muitas características externas ao texto podem servir para descriminar se
uma pessoa está ou não sendo sarcástica. Entre alguns exemplos estão intonação,
locutor, interlocutor, conhecimento prévio sobre a fala, tempo e espaço em que
se fala e elementos não verbais.

(Humans Require Context to Infer Ironic Intent (so Computers Probably do, too))

O problema de detecção de sarcasmo pode ser tratado como uma classificação
binária. Esse tipo de problema é muito estudado no campo do aprendizado de
máquina e envolve classificar os elementos de um conjunto em dois grupos
chamados de classes. Por classificar entende-se a determinação de um elemento do
conjunto entre pertencente à primeira ou segunda classe.

No caso da detecção de sarcasmo, os elementos são os pedaços de texto e as duas
classes são \textit{ser sarcástico} e \textit{não ser sarcástico}. Em termos
matemáticos, dizemos que temos um conjunto de exemplos denotado pela matriz $X$
de dimensões $n\times m$ onde $n$ é o número de exemplos e $m$ é o número de
características que usamos para descrever cada um desses exemplos. Cada linha
de $X$ é denotada por $\xii[x]{i}$ e chamada de \textit{exemplo} $i$ ou
\textit{instância} $i$.

Cada instância $i$ pertence ou à primeira classe ou à segunda, modelamos isso
por um valor $\xii[y]{i}$ chamado de $i$-ésimo rótulo (do inglês,
\textit{label}). E dizemos que $\xii[y]{i}\in\set{0, 1}$, onde cada valor
representa uma das classes. No caso de sarcasmo, podemos representar por
\textit{não sarcasmo} o valor $0$ e \textit{sarcasmo}, o valor $1$. Com esses
valores $\xii[y]{i}$ construímos um vetor $y$ onde $y_i=\xii[y]{i}$.

Ao final, o problema é descobrir uma função $f$ que mapeie bem $X$ em $y$. Note
que $n$, o número de exemplos, é possivelmente infinito, pois sempre se pode
achar novos exemplos. Ou seja, tenta-se achar essa função $f$ conhecendo
parcialmente o conjunto das instâncias.

\section{Trabalhos Relacionados}%
\label{sec:trabalhos_relacionados}


