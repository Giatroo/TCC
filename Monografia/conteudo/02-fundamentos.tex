%!TeX root=../tese.tex

%%%%%%%%%%%%%%%%%%%%%%%%%%%%%%%%%%%%%%%%%%%%%%%%%%%%%%%%%%%%%%%%%%%%%%%%%%%%%%%%

\chapter{Fundamentos e Trabalhos Relacionados}%
\label{cha:fundamentos_e_trabalhos_relacionados}

\section{Detecção de Sarcasmo}%
\label{sec:deteccao_de_sarcasmo}

De acordo com o dicionário Dicio, sarcasmo é uma zombaria que busca ofender,
enquanto ironia é a ação de dizer o oposto do que se deseja expressar. Ainda
segundo ele, a diferença entre esses dois termos se dá no fato de que sarcasmo é
um dito ácido que pode ou não ser expresso por meio de uma ironia e ironia, por
sua vez, pode ou não ser utilizada para ofender.~\cite{dicio_sarc, dicio_irony}
(H. Paul Grice. 1975. Logic and Conversation)
(Irony and Sarcasm: Corpus Generation and Analysis Using Crowdsourcin)

O termo detecção de sarcasmo, por sua vez, refere-se à determinação de se há ou
não sarcasmo em uma porção de texto verbal.

Em geral, esse problema é difícil, pois, por vezes, nem o interlocutor consegue
perceber essa figura de linguagem. Além disso, o contexto tende a importar
muito.





