%!TeX root=../tese.tex
%("dica" para o editor de texto: este arquivo é parte de um documento maior)
% para saber mais: https://tex.stackexchange.com/q/78101/183146

% Insira aqui os metadados do seu trabalho. Para isso, copie,
% com as alterações necessárias, o conteúdo do arquivo
% conteudo-exemplo/metadados.tex

\title{
% Obrigatório nas duas línguas
titlept={Detecção de sarcasmo em redes sociais utilizando DeBERTa},
titleen={Sarcasm detection in social media using decoding-enhanced BERT with
disentangled attention},
% Opcional, mas se houver deve existir nas duas línguas
%subtitlept={um subtítulo},
%subtitleen={a subtitle},
% Opcional, para o cabeçalho das páginas
% shorttitle={Detecção de sarcasmo em redes sociais},
}

\author{Lucas Paiolla Forastiere}

% Para TCCs, este comando define o supervisor
\orientador{Prof. Dr. Ricardo Marcondes Marcacini}

\defesa{
  nivel=tcc, % mestrado, doutorado ou tcc
  programa={Ciência da Computação},
  local={São Paulo},
  data=2022-12-23, % YYYY-MM-DD
  % A licença do seu trabalho. Use CC-BY, CC-BY-NC, CC-BY-ND, CC-BY-SA,
  % CC-BY-NC-SA ou CC-BY-NC-ND para escolher a licença Creative Commons
  % correspondente (o sistema insere automaticamente o texto da licença).
  % Se quiser estabelecer regras diferentes para o uso de seu trabalho,
  % converse com seu orientador e coloque o texto da licença aqui, mas
  % observe que apenas TCCs sob alguma licença Creative Commons serão
  % acrescentados ao BDTA.
  direitos={CC-BY}, % Creative Commons Attribution 4.0 International License
  %direitos={Autorizo a reprodução e divulgação total ou parcial
  %          deste trabalho, por qualquer meio convencional ou
  %          eletrônico, para fins de estudo e pesquisa, desde que
  %          citada a fonte.},
  % Isto deve ser preparado em conjunto com o bibliotecário
  %fichacatalografica={nome do autor, título, etc.},
}

% As palavras-chave são obrigatórias, em português e
% em inglês. Acrescente quantas forem necessárias.
\palavrachave{Inteligência Artificial}
\palavrachave{Aprendizado de Máquina}
\palavrachave{Processamento de Linguagem Natural}
\palavrachave{Aprendizado Profundo}
\palavrachave{Detecção de Sarcasmo}

\keyword{Artificial Inteligence}
\keyword{Machine Learning}
\keyword{Natural Language Processing}
\keyword{Deep Learning}
\keyword{Sarcasm Detection}

% O resumo é obrigatório, em português e inglês.
\resumo{
Detecção de sarcasmo automática é um importante fator de muitos sistemas de
compreenção de linguagem natural. Ela é uma tarefa difícil tanto para seres
humanos como para máquinas, pois requer que o leitor possua, muitas vezes,
informações contextuais específicas a respeito do que se fala. Este trabalho
aplica um modelo de aprendizado profundo do tipo \textit{transformer} chamado
DeBERTa no conjunto de dados SARC, com o objetivo de detectar respostas
sarcásticas em comentários feitos na rede social Reddit. Os experimentos
comparar o DeBERTa com outros modelos do tipo \textit{transformers}, como o BERT
e o RoBERTa, e encontra uma melhora nas métricas em relação a estes modelos.
}

\abstract{
Automatic sarcasm detection is an important part of many natural language
understanding systems. It is a hard task both for humans and machines, because,
many times, it requires the reader to have specific contextual information about
what is being talked about. This paper applies a transformer deep learning model
called DeBERTa on the SARC dataset, to detect sarcastic answers on comments made
on the Reddit social media. The experiments compare DeBERTa with others
transformers, like BERT and RoBERTa, and find an improvement on metrics in
respect to those models.
}
