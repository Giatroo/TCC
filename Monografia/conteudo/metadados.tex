%!TeX root=../tese.tex
%("dica" para o editor de texto: este arquivo é parte de um documento maior)
% para saber mais: https://tex.stackexchange.com/q/78101/183146

% Insira aqui os metadados do seu trabalho. Para isso, copie,
% com as alterações necessárias, o conteúdo do arquivo
% conteudo-exemplo/metadados.tex

\title{
% Obrigatório nas duas línguas
titlept={Detecção de sarcasmo em redes sociais utilizando DeBERTa},
titleen={Sarcasm detection in social media using decoding-enhanced BERT with
disentangled attention},
% Opcional, mas se houver deve existir nas duas línguas
%subtitlept={um subtítulo},
%subtitleen={a subtitle},
% Opcional, para o cabeçalho das páginas
% shorttitle={Detecção de sarcasmo em redes sociais},
}

\author{Lucas Paiolla Forastiere}

% Para TCCs, este comando define o supervisor
\orientador{Prof. Dr. Ricardo Marcondes Marcacini}

\defesa{
  nivel=tcc, % mestrado, doutorado ou tcc
  programa={Ciência da Computação},
  local={São Paulo},
  data=2022-12-23, % YYYY-MM-DD
  % A licença do seu trabalho. Use CC-BY, CC-BY-NC, CC-BY-ND, CC-BY-SA,
  % CC-BY-NC-SA ou CC-BY-NC-ND para escolher a licença Creative Commons
  % correspondente (o sistema insere automaticamente o texto da licença).
  % Se quiser estabelecer regras diferentes para o uso de seu trabalho,
  % converse com seu orientador e coloque o texto da licença aqui, mas
  % observe que apenas TCCs sob alguma licença Creative Commons serão
  % acrescentados ao BDTA.
  direitos={CC-BY}, % Creative Commons Attribution 4.0 International License
  %direitos={Autorizo a reprodução e divulgação total ou parcial
  %          deste trabalho, por qualquer meio convencional ou
  %          eletrônico, para fins de estudo e pesquisa, desde que
  %          citada a fonte.},
  % Isto deve ser preparado em conjunto com o bibliotecário
  %fichacatalografica={nome do autor, título, etc.},
}

% As palavras-chave são obrigatórias, em português e
% em inglês. Acrescente quantas forem necessárias.
\palavrachave{Palavra-chave1}
\palavrachave{Palavra-chave2}
\palavrachave{Palavra-chave3}

\keyword{Keyword1}
\keyword{Keyword2}
\keyword{Keyword3}

% O resumo é obrigatório, em português e inglês.
\resumo{
Elemento obrigatório, constituído de uma sequência de frases concisas e
objetivas, em forma de texto.  Deve apresentar os objetivos, métodos empregados,
resultados e conclusões.  O resumo deve ser redigido em parágrafo único, conter
no máximo 500 palavras e ser seguido dos termos representativos do conteúdo do
trabalho (palavras-chave). Deve ser precedido da referência do documento.
Texto texto texto texto texto texto texto texto texto texto texto texto texto
texto texto texto texto texto texto texto texto texto texto texto texto texto
texto texto texto texto texto texto texto texto texto texto texto texto texto
texto texto texto texto texto texto texto texto texto texto texto texto texto
texto texto texto texto texto texto texto texto texto texto texto texto texto
texto texto texto texto texto texto texto texto.
Texto texto texto texto texto texto texto texto texto texto texto texto texto
texto texto texto texto texto texto texto texto texto texto texto texto texto
texto texto texto texto texto texto texto texto texto texto texto texto texto
texto texto texto texto texto texto texto texto texto texto texto texto texto
texto texto.
}

\abstract{
Elemento obrigatório, elaborado com as mesmas características do resumo em
língua portuguesa. De acordo com o Regimento da Pós-Graduação da USP (Artigo
99), deve ser redigido em inglês para fins de divulgação. É uma boa ideia usar
o sítio \url{www.grammarly.com} na preparação de textos em inglês.
Text text text text text text text text text text text text text text text text
text text text text text text text text text text text text text text text text
text text text text text text text text text text text text text text text text
text text text text text text text text text text text text.
Text text text text text text text text text text text text text text text text
text text text text text text text text text text text text text text text text
text text text.
}
