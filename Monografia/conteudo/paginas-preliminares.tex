%!TeX root=../tese.tex
%("dica" para o editor de texto: este arquivo é parte de um documento maior)
% para saber mais: https://tex.stackexchange.com/q/78101/183146

% Apague as duas linhas abaixo (elas servem apenas para gerar um
% aviso no arquivo PDF quando não há nenhum dado a imprimir) e
% copie, com as alterações necessárias, o conteúdo do arquivo
% conteudo-exemplo/paginas-preliminares.tex

%%%%%%%%%%%%%%%%%%%% DEDICATÓRIA, RESUMO, AGRADECIMENTOS %%%%%%%%%%%%%%%%%%%%%%%

% Resumo e abstract são definidos no arquivo "metadados.tex". Este
% comando também gera automaticamente a referência para o próprio
% documento, conforme as normas sugeridas da USP.
\printResumoAbstract

%%%%%%%%%%%%%%%%%%%%%%%%%%% LISTAS DE FIGURAS ETC. %%%%%%%%%%%%%%%%%%%%%%%%%%%%%

% Como as listas que se seguem podem não incluir uma quebra de página
% obrigatória, inserimos uma quebra manualmente aqui.
\makeatletter
\if@openright\cleardoublepage\else\clearpage\fi
\makeatother

% Todas as listas são opcionais; Usando "\chapter*" elas não são incluídas
% no sumário. As listas geradas automaticamente também não são incluídas por
% conta das opções "notlot" e "notlof" que usamos para a package tocbibind.

% Normalmente, "\chapter*" faz o novo capítulo iniciar em uma nova página, e as
% listas geradas automaticamente também por padrão ficam em páginas separadas.
% Como cada uma destas listas é muito curta, não faz muito sentido fazer isso
% aqui, então usamos este comando para desabilitar essas quebras de página.
% Se você preferir, comente as linhas com esse comando e des-comente as linhas
% sem ele para criar as listas em páginas separadas. Observe que você também
% pode inserir quebras de página manualmente (com \clearpage, veja o exemplo
% mais abaixo).
\newcommand\disablenewpage[1]{{\let\clearpage\par\let\cleardoublepage\par #1}}

% Nestas listas, é melhor usar "raggedbottom" (veja basics.tex). Colocamos
% a opção correspondente e as listas dentro de um grupo para ativar
% raggedbottom apenas temporariamente.
\bgroup
\raggedbottom

%%%%% Listas criadas manualmente

%\chapter*{Lista de Abreviaturas}
\disablenewpage{\chapter*{Lista de Abreviaturas}}

\begin{tabular}{rl}
PLN & Processamento de linguagem natural (\emph{Natural language processing})
\\
API & Interface de programação de aplicação (\emph{Application programming
interface}) \\
DT & Árvore de decisão (\emph{Decision tree}) \\
RF & Floresta aleatória (\emph{Random forest}) \\
SVM & - (\emph{Support vector machine}) \\
NB & - (\emph{naive Bayes}) \\
NN & Rede Neural (\emph{Neural Network}) \\
LSTM & - (\emph{Long short-term memory}) \\
RNN & Rede neural recorrente (\emph{Recurrent neural network}) \\
CNN & Rede neural convolucional (\emph{Convolutional neural network}) \\
DNN & Rede neural profunda (\emph{Deep neural network}) \\
BoW & Sacola de palavras (\emph{Bag-of-words}) \\
Tf-Idf & - (\emph{Term frequency-inverse document frequency}) \\
SES & Suavização exponencial simples (\emph{Simple exponential smoothing}) \\
MLM & Modelo de linguagem mascarada (\emph{Masked language model}) \\
RTD & Detecção de símbolo substituído (\emph{Replaced token detection}) \\
URL & Localizador Uniforme de Recursos (\emph{Uniform Resource Locator})\\
ASCII & Código padrão americano para o intercâmbio de informação (\emph{American
Standard} \\
      & \emph{Code for Information Interchange}) \\
IME & Instituto de Matemática e Estatística \\
USP & Universidade de São Paulo
\end{tabular}

% Quebra de página manual
\clearpage

%\chapter*{Lista de Símbolos}
\disablenewpage{\chapter*{Lista de Símbolos}}

\begin{tabular}{rl}
$X$ & Características de entrada\\
$y$ & Rótulos de saída\\
$\hat{y}$ & Rótulos preditos de saída\\
$tp$ & Quantidade de verdadeiros positivos (\emph{True positives})\\
$fp$ & Quantidade de falsos positivos (\emph{False positives})\\
$tn$ & Quantidade de verdadeiros negativos (\emph{True negatives})\\
$fn$ & Quantidade de falsos negativos (\emph{False negatives})\\
$F_1$ & Métrica F1\\
$\xii[x]{i}$ & $i$-ésimo exemplo de entrada\\
$\xii[y]{i}$ & $i$-ésimo exemplo de saída\\
$\text{exp}$ & função exponencial\\
$Q$ & Consulta (\emph{query}) do mecanismo de atenção\\
$K$ & Chave (\emph{key}) do mecanismo de atenção\\
$V$ & Valor (\emph{value}) do mecanismo de atenção\\
\texttt{<MASK>} & Símbolo para o \textit{token} especial de máscara\\
\end{tabular}

% Quebra de página manual
\clearpage

%%%%% Listas criadas automaticamente

% Você pode escolher se quer ou não permitir a quebra de página
%\listoffigures
\disablenewpage{\listoffigures}

% Você pode escolher se quer ou não permitir a quebra de página
%\listoftables
\disablenewpage{\listoftables}

% Sumário (obrigatório)
\tableofcontents

\egroup % Final de "raggedbottom"

