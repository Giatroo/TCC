%!TeX root=../tese.tex
%("dica" para o editor de texto: este arquivo é parte de um documento maior)
% para saber mais: https://tex.stackexchange.com/q/78101/183146

%%%%%%%%%%%%%%%%%%%%%%%%%%%%%%%%%%%%%%%%%%%%%%%%%%%%%%%%%%%%%%%%%%%%%%%%%%%%%%%%
%%%%%%%%%%%%%%%%%%%%%%%%%%%%% METADADOS DA TESE %%%%%%%%%%%%%%%%%%%%%%%%%%%%%%%%
%%%%%%%%%%%%%%%%%%%%%%%%%%%%%%%%%%%%%%%%%%%%%%%%%%%%%%%%%%%%%%%%%%%%%%%%%%%%%%%%

% Estes comandos definem o título e autoria do trabalho e devem sempre ser
% definidos, pois além de serem utilizados para criar a capa, também são
% armazenados nos metadados do PDF.
\title{
    % Obrigatório nas duas línguas
    titlept={Título do trabalho},
    titleen={Title of the document},
    % Opcional, mas se houver deve existir nas duas línguas
    subtitlept={um subtítulo},
    subtitleen={a subtitle},
    % Opcional, para o cabeçalho das páginas
    shorttitle={Título curto},
}

\author[fem]{Nome Completo}

% Para TCCs, este comando define o supervisor
\orientador[fem]{Profª. Drª. Fulana de Tal}

% Se não houver, remova; se houver mais de um, basta
% repetir o comando quantas vezes forem necessárias
\coorientador{Prof. Dr. Ciclano de Tal}
\coorientador[fem]{Profª. Drª. Beltrana de Tal}

% A página de rosto da versão para depósito (ou seja, a versão final
% antes da defesa) deve ser diferente da página de rosto da versão
% definitiva (ou seja, a versão final após a incorporação das sugestões
% da banca).
\defesa{
  nivel=mestrado, % mestrado, doutorado ou tcc
  % É a versão para defesa ou a versão definitiva?
  %definitiva,
  % É qualificação?
  %quali,
  programa={Ciência da Computação},
  membrobanca={Profª. Drª. Fulana de Tal (orientadora) -- IME-USP [sem ponto final]},
  % Em inglês, não há o "ª"
  %membrobanca{Prof. Dr. Fulana de Tal (advisor) -- IME-USP [sem ponto final]},
  membrobanca={Prof. Dr. Ciclano de Tal -- IME-USP [sem ponto final]},
  membrobanca={Profª. Drª. Convidada de Tal -- IMPA [sem ponto final]},
  % Se não houve bolsa, remova
  %
  % Norma sobre agradecimento por auxílios da FAPESP:
  % https://fapesp.br/11789/referencia-ao-apoio-da-fapesp-em-todas-as-formas-de-divulgacao
  %
  % Norma sobre agradecimento por auxílios da CAPES (Portaria 206,
  % de 4 de Setembro de 2018):
  % https://www.in.gov.br/materia/-/asset_publisher/Kujrw0TZC2Mb/content/id/39729251/do1-2018-09-05-portaria-n-206-de-4-de-setembro-de-2018-39729135
  %
  %apoio={O presente trabalho foi realizado com apoio da Coordenação
  %       de Aperfeiçoamento\\ de Pessoal de Nível Superior -- Brasil
  %       (CAPES) -- Código de Financiamento 001}, % o código é sempre 001
  %
  %apoio={This study was financed in part by the Coordenação de
  %       Aperfeiçoamento\\ de Pessoal de Nível Superior -- Brasil
  %       (CAPES) -- Finance Code 001}, % o código é sempre 001
  %
  %apoio={Durante o desenvolvimento deste trabalho, o autor recebeu\\
  %       auxílio financeiro da FAPESP -- processo nº aaaa/nnnnn-d},
  %
  %apoio={During the development if this work, the author received\\
  %       financial support from FAPESP -- grant \#aaaa/nnnnn-d},
  %
  apoio={Durante o desenvolvimento deste trabalho o autor
         recebeu auxílio financeiro da XXXX},
  local={São Paulo},
  data=2017-08-10, % YYYY-MM-DD
  % A licença do seu trabalho. Use CC-BY, CC-BY-NC, CC-BY-ND, CC-BY-SA,
  % CC-BY-NC-SA ou CC-BY-NC-ND para escolher a licença Creative Commons
  % correspondente (o sistema insere automaticamente o texto da licença).
  % Se quiser estabelecer regras diferentes para o uso de seu trabalho,
  % converse com seu orientador e coloque o texto da licença aqui, mas
  % observe que apenas TCCs sob alguma licença Creative Commons serão
  % acrescentados ao BDTA.
  direitos={CC-BY}, % Creative Commons Attribution 4.0 International License
  %direitos={Autorizo a reprodução e divulgação total ou parcial
  %          deste trabalho, por qualquer meio convencional ou
  %          eletrônico, para fins de estudo e pesquisa, desde que
  %          citada a fonte.},
  % Isto deve ser preparado em conjunto com o bibliotecário
  %fichacatalografica={nome do autor, título, etc.},
}

% As palavras-chave são obrigatórias, em português e
% em inglês. Acrescente quantas forem necessárias.
\palavrachave{Palavra-chave1}
\palavrachave{Palavra-chave2}
\palavrachave{Palavra-chave3}

\keyword{Keyword1}
\keyword{Keyword2}
\keyword{Keyword3}

% O resumo é obrigatório, em português e inglês.
\resumo{
Elemento obrigatório, constituído de uma sequência de frases concisas e
objetivas, em forma de texto.  Deve apresentar os objetivos, métodos empregados,
resultados e conclusões.  O resumo deve ser redigido em parágrafo único, conter
no máximo 500 palavras e ser seguido dos termos representativos do conteúdo do
trabalho (palavras-chave). Deve ser precedido da referência do documento.
Texto texto texto texto texto texto texto texto texto texto texto texto texto
texto texto texto texto texto texto texto texto texto texto texto texto texto
texto texto texto texto texto texto texto texto texto texto texto texto texto
texto texto texto texto texto texto texto texto texto texto texto texto texto
texto texto texto texto texto texto texto texto texto texto texto texto texto
texto texto texto texto texto texto texto texto.
Texto texto texto texto texto texto texto texto texto texto texto texto texto
texto texto texto texto texto texto texto texto texto texto texto texto texto
texto texto texto texto texto texto texto texto texto texto texto texto texto
texto texto texto texto texto texto texto texto texto texto texto texto texto
texto texto.
}

\abstract{
Elemento obrigatório, elaborado com as mesmas características do resumo em
língua portuguesa. De acordo com o Regimento da Pós-Graduação da USP (Artigo
99), deve ser redigido em inglês para fins de divulgação. É uma boa ideia usar
o sítio \url{www.grammarly.com} na preparação de textos em inglês.
Text text text text text text text text text text text text text text text text
text text text text text text text text text text text text text text text text
text text text text text text text text text text text text text text text text
text text text text text text text text text text text text.
Text text text text text text text text text text text text text text text text
text text text text text text text text text text text text text text text text
text text text.
}
